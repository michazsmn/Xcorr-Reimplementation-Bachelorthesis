\documentclass{article}
\usepackage{biblatex}
\usepackage{graphicx} % Required for inserting images

\addbibresource{bibliography.bib}

\title{Xcorr Reimplementation Bachelorthesis}
\author{Michael zusmanovskiy}
\date{November 2024}

\begin{document}

\begin{titlepage}
    \begin{center}
        \LARGE
        Applied Computer Science

        \vspace{1.5cm}
            
        \LARGE
        \textbf{Re-implementation of the X-Corr algorithm in Python and improvement of scoring using predicted spectra}
            
        \vspace{1.5cm}

        \LARGE
        \textbf{Author}
        
        Michael Zusmanovskiy, 108019231182
        
        \vspace{1.5cm}
        
        \LARGE
        \textbf{Supervisors}
        
        Jun.-Prof. Julian Uszkoreit
        
        M.Sc. Dirk Winkelhardt
            
        \vfill
            

        \vspace{0.8cm}

        \Large
        Ruhr-Universität Bochum\\
        
        Submission date: 30. 12. 2024
            
    \end{center}
\end{titlepage}


\section{Introduction}
\subsection{Motivation}
\(MS^2\) Database searching is a method for peptide identification within a Mass Spectrometry Scan. Proteomics software
that deals with the problem of identifying protein/peptide identification can fall into the category of database searching,
where the search is performed against a database, and de novo searching, where peptides are deduced without a database.

Comet is an open source proteomics database search algorithm/tool for searching spectrum data against sequence databases and identification of peptides\cite{comet}


\subsection{Goals}

\section{Methods}

\section{Discussion and Conclusion}
\subsection{Evaluation}
\subsection{Future Work}
\subsection{Impact \& Conclusion}


\printbibliography

\end{document}
